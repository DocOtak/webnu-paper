\documentclass[12pt,a4paper]{article}
\usepackage[utf8]{inputenc}
\usepackage{amsmath}
\usepackage{amsfonts}
\usepackage{amssymb}
\usepackage{graphicx}
\usepackage{setspace}
\usepackage{hyperref}
\usepackage[left=2cm,right=2cm,top=2cm,bottom=2cm]{geometry}
\usepackage{authblk}

\title{Neutrino Geoscience with Real-time Modeling in the Web Browser}
\author[1]{A. M. Barna}
\author[2]{G. R. Jocher}
\author[3]{S. T. Dye}
\affil[1]{Scripps Institution of Oceanography 9500 Gilman Drive La Jolla, California 92093-0214}
\affil[2]{Ultralytics LLC, Arlington, VA 22201}
\affil[3]{Hawaii Pacific University 45-045 Kamehameha Hwy, Kaneohe, HI 96744}
\renewcommand\Affilfont{\itshape\tiny}

%%%%%%%%%%%%%%%%%%%%%%%%%%%%%
% Is tracked with version   %
% control software, please  %
% use one sentence per line %
%%%%%%%%%%%%%%%%%%%%%%%%%%%%%

\begin{document}
\maketitle
\begin{abstract}
  This is just the AGU submitted abstract... fix when done with paper.

  We present a real-time, interactive, web-based visualization of the Earth's antineutrino flux from both natural and man-made sources.
  The initial phase of development displays crust layer thicknesses, surface heat flux due to heat-producing elements, uranium, thorium and potassium, and the surface geo-neutrino flux from these elements in the mantle and crust.
  It permits users to define heat-producing element concentrations in the mantle, including a global constraint on the radiogenic power and the Th/U and K/U ratios of the bulk silicate Earth.
  Moreover, users have various output visualization options for geo-neutrinos, including flux or detectable signal from relevant heat-producing elements.
  The mantle signal is available in several forms as a function of the background from the crust and nuclear reactors.
  Our goal is to implement a tool for education and outreach as well as the research community.
\end{abstract}
\clearpage

\section{Introduction}
Earth scientists face a deceptivly simple problem, what is the earth made of and where is that material located.
Related questions include the origin of earths heat and the distributions of the major radiogenic elements \textsuperscript{40}K, \textsuperscript{232}U, and \textsuperscript{238}U.
The historic tools available to provide insight are siesmic profiles, cosmologic comparisons, and laboratory modeling \cite{McDonough1995}.
More recently, geoneutrinos are offering the ability to measure a process occuring within the earth\cite{McDonough2012}.

Neutrinos are a product of beta decay with a property lending them to be incredibly useful for measuring nuclear processes within the earth, matter is nearly transparent to them.
Indeed, the probability of an interaction between a single particle and a light-years thickness of lead is around 50 percent\cite{McDonough2014}.
The very property making neutrinos useful for observing interior earth nuclear decay also makes them dificult to detect, requiring kiloton detectors for just tens of geoneutrinos interactions per year.

Radiogenic decay within the mantle are not the only source of neutrinos.
A significant amount of the neutrinos detected are from more local crust and nuclear reactor sources.
To measure the mantle signal, the other sources of neutrinos must be charicterized and accounted for and are the source of the largest component of the uncertanty\cite{Dye2010}.
The high uncertainty in geoneutrino measurements allows for the geochemical, geophysical and cosomological models, only the fully radiologic heating model has been excluded with any certainty\cite{McDonough2012}.

In this paper, we will present initial work on a simplified, web browser based, interactive earth model.
Our model allows for quick visualization of several preset mantle models or any user input model.
The model includes crust and reactor sources and allows for the display of several output parameters.
It is our hope it becomes a useful tool for research as well as education.

\section{Methods}
This section will discuss methods in several parts.
Data sources include the mantle and crust input models as well as some precomputed data.
The technologies will contain the libraries and more recent improvements in web browsers making the visualizations possible.
Finally the data model used on the user agents as well as how user input is processed will be dicussed.

\subsection{Data sources}
\subsubsection{PREM}
The base mantle model is the preliminary reference earth model (PREM), a siesmological model of the entire earth.
PREM is divided into 200km thick concentric layers with additional breaks for major discontinuities within the mnatle.
Only the silicate earth portions of PREM are used in our model, the core is assumed to contain negligible radiogenic elements.
PREM is used for all mantle calculations, all layers between the LID and the core mantle boundary allow for user input.

TODO Needs Citations for PREM
\subsubsection{CRUST 2.0}
The base crust model used is CRUST 2.0.
CRUST 2.0 is a 2 by 2 degree earth crust model further divided into 7 layers.
Those 7 layers include surface ice, the liqud hydrosphere, soft and hard sediments, and an upper, middle and lower crusts.
The model provided density and thickness which are both used to calculate mass.
The mass, area, and ocean flags were precomputed and converted to a json format, to be described later.

TODO Needs citations for CRUST2.0
\subsubsection{Precomputed Neutrino Fluxes}
TODO!!!

The data are from Huang et. al.
Were provided in 1 by 1 degree bins, used area weighted mean to convert to 2 by 2 degree bins
Were converted to TNU for each element (K, Th, and U)
Converted to json

\subsubsection{Reactor Data}
TODO!!!

Data are from Jocher,
Were provided in 1 by 1 degree bins, used area weighted mean to convert to 2 by 2 degree bins
converted to json

\subsection{Web Technology}
The platform independet nature of web browsers makes it ideal for developing websites which go beyond simple static content, approaching the ability to act as full applications.
Several technologies make interactive web based applications possible, to name a few: ECMAScript, CSS, HTML5 Canvas, SVG, and HTML (HTML5 specifically).
HTML is the basic document used by web browsers for rendering, it contains information on what other content (scripts, images, styles) should be loaded, and usually has the majority of the text content for a web page.
Cascading Style Sheets (CSS) contain layout information on how the HTML should be rendered and layed out in a web browser.
Scalabale Vector Graphics (SVG) provide a cross platform format for displaying vector graphics.
HTML5 Canvas is a recent addition to the HTML standard which allows for browser based scripts to draw raster graphics to a defined area on the webpage.
Finally, ECMAScript 5, commonly known as JavaScript (JS) or JScript, allows for computation to be performed within the web browser as well as interact with elements on a web page.
In addition to the browser based technologies, preprocessing of data was performed using the Python scripting language.
The specifics of how each of these technologies was used will be discussed in this section.

\subsubsection{Python}
Python has grown in popularity within many scientific communities recently, largly due to ease of use and the many scientific libraries available for it.
It is an interpreted, object orientated scripting language with a MatLab like syntax.
Python was used to prepare input data such as the crust neutrino fluxes and signals.
It was used to convert the largly tabular data into the JSON format which the browser based JS interpreter can understand.
Initial implimentations of geoneutrinos.org relied entirely on real time server side processing done using python.
However, it was found python was not fast enough to provide the impression of responsiveness needed for real time interaction.
The computations were shifted to the client side (web browser) to reduce both server load and to provide more real time visualization computations.

\subsubsection{HTML5 and CSS}
Talk about HTML, the DOM, what CSS does with it.

Talk about twitter bootstrap as the UI layout library and what we get with it (support for multiple screen widths, as small as cell phones, for almost free)

\subsubsection{JavaScript and JavaScript Libraries}
Talk about how JS interacts with the DOM

Talk about what we get from the D3 library

Talk about what we get with jQuery

\subsubsection{Canvas and SVG}
Talk about how we display the map

Talk about how we display the color bar

Talk about how we display the continents and plate boundary overlays.


TODO: Remove this list, only here as a reminder of what to write about.
\begin{itemize}
  \item Languages used:
  \begin{itemize}
    \item Python
    \item JavaScript
  \end{itemize}
  \item  Libraries Used
  \begin{itemize}
    \item D3
    \item JQuery
    \item Bootstrap
  \end{itemize}
  \item Modern Web Technologies:
  \begin{itemize}
    \item SVG (Color bars, map overlays)
    \item HTML5 Canvas (Actual Map)
  \end{itemize}
\end{itemize}

\subsection{Data Model and User Input}
Internal data model:

What information is persisted and how.

How JS handles numberic data

How we use the JS objects to store things

How we perform math on 2d arrays (not native to JavaScript)

Event based nature of web browsers and what happens when a user moves a control

The maths done when user inputs.

\section{Discussion}
Discuss potential researcher benifit.

Discuss potential education benifit.

Discuss future (obligatory additional work required)
\section{Source Code}
The source for this document as well as the website it describes is freely available on GitHub.
\begin{itemize}
 \item Website: \url{https://github.com/docotak/geoneutrinos.org}
 \item Document: TODO: push this to github
\end{itemize}
\section{Acknowledgments}
HPU, CIDER
\bibliographystyle{plain}
\bibliography{refs}
\end{document}

