\documentclass[12pt,a4paper]{article}
\usepackage[utf8]{inputenc}
\usepackage{amsmath}
\usepackage{amsfonts}
\usepackage{amssymb}
\usepackage{graphicx}
\usepackage{setspace}
\usepackage[left=2cm,right=2cm,top=2cm,bottom=2cm]{geometry}
\usepackage{authblk}

\title{Neutrino Geoscience with Real-time Modeling in the Web Browser}
\author[1]{A. M. Barna}
\author[2]{G. R. Jocher}
\author[3]{S. T. Dye}
\affil[1]{Scripps Institution of Oceanography 9500 Gilman Drive La Jolla, California 92093-0214}
\affil[2]{Ultralytics LLC, Arlington, VA 22201}
\affil[3]{Hawaii Pacific University 45-045 Kamehameha Hwy, Kaneohe, HI 96744}
\renewcommand\Affilfont{\itshape\tiny}

\begin{document}
\maketitle
\begin{abstract}
  We present a real-time, interactive, web-based visualization of the Earth's antineutrino flux from both natural and man-made sources.
  The initial phase of development displays crust layer thicknesses, surface heat flux due to heat-producing elements, uranium, thorium and potassium, and the surface geo-neutrino flux from these elements in the mantle and crust.
  It permits users to define heat-producing element concentrations in the mantle, including a global constraint on the radiogenic power and the Th/U and K/U ratios of the bulk silicate Earth.
  Moreover, users have various output visualization options for geo-neutrinos, including flux or detectable signal from relevant heat-producing elements.
  The mantle signal is available in several forms as a function of the background from the crust and nuclear reactors.
  Our goal is to implement a tool for education and outreach as well as the research community.
\end{abstract}
\clearpage

\section{Introduction}
\section{Methods}
%\bibliographystyle{plain}
%\bibliography{refs}
\end{document}

