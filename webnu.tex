\documentclass[12pt,a4paper]{article}
\usepackage[utf8]{inputenc}
\usepackage{amsmath}
\usepackage{amsfonts}
\usepackage{amssymb}
\usepackage{graphicx}
\usepackage{setspace}
\usepackage[left=2cm,right=2cm,top=2cm,bottom=2cm]{geometry}
\usepackage{authblk}

\title{Neutrino Geoscience with Real-time Modeling in the Web Browser}
\author[1]{A. M. Barna}
\author[2]{G. R. Jocher}
\author[3]{S. T. Dye}
\affil[1]{Scripps Institution of Oceanography 9500 Gilman Drive La Jolla, California 92093-0214}
\affil[2]{Ultralytics LLC, Arlington, VA 22201}
\affil[3]{Hawaii Pacific University 45-045 Kamehameha Hwy, Kaneohe, HI 96744}
\renewcommand\Affilfont{\itshape\tiny}

%%%%%%%%%%%%%%%%%%%%%%%%%%%%%
% Is tracked with version   %
% control software, please  %
% use one sentence per line %
%%%%%%%%%%%%%%%%%%%%%%%%%%%%%

\begin{document}
\maketitle
\begin{abstract}
  This is just the AGU submitted abstract... fix when done with paper.

  We present a real-time, interactive, web-based visualization of the Earth's antineutrino flux from both natural and man-made sources.
  The initial phase of development displays crust layer thicknesses, surface heat flux due to heat-producing elements, uranium, thorium and potassium, and the surface geo-neutrino flux from these elements in the mantle and crust.
  It permits users to define heat-producing element concentrations in the mantle, including a global constraint on the radiogenic power and the Th/U and K/U ratios of the bulk silicate Earth.
  Moreover, users have various output visualization options for geo-neutrinos, including flux or detectable signal from relevant heat-producing elements.
  The mantle signal is available in several forms as a function of the background from the crust and nuclear reactors.
  Our goal is to implement a tool for education and outreach as well as the research community.
\end{abstract}
\clearpage

\section{Introduction}
Earth scientists face a deceptivly simple problem, what is the earth made of and where is that material located.
Related questions include the origin of earths heat and the distributions of the major radiogenic elements \textsuperscript{40}K, \textsuperscript{232}U, and \textsuperscript{238}U.
The historic tools available to provide insight are siesmic profiles, cosmologic comparisons, and laboratory modeling \cite{McDonough1995}.
More recently, geoneutrinos are offering the ability to measure a process occuring within the earth\cite{McDonough2012}.

Neutrinos are a product of beta decay with a property lending them to be incredibly useful for measuring nuclear processes within the earth, matter is nearly transparent to them.
Indeed, the probability of an interaction between a single particle and a light-years thickness of lead is around 50 percent\cite{McDonough2014}.
The very property making neutrinos useful for observing interior earth nuclear decay also makes them dificult to detect, requiring kiloton detectors for just tens of geoneutrinos interactions per year.

Radiogenic decay within the mantle are not the only source of neutrinos.
A significant amount of the neutrinos detected are from more local crust and nuclear reactor sources.
To measure the mantle signal, the other sources of neutrinos must be charicterized and accounted for and are the source of the largest component of the uncertanty\cite{Dye2010}.
The high uncertainty in geoneutrino measurements allows for the geochemical, geophysical and cosomological models, only the fully radiologic heating model has been excluded with any certainty\cite{McDonough2012}.

In this paper, we will present initial work on a simplified, web browser based, interactive earth model.
Our model allows for quick visualization of several preset mantle models or any user input model.
The model includes crust and reactor sources and allows for the display of several output parameters.
It is our hope it becomes a useful tool for research as well as education.

\section{Methods}
Data sources:
  \begin{itemize}
    \item Dye \cite{Dye2012}
    \item Jocher (reactor flux)
    \item Huang et. al. 2013 (Neutrino fluxes)?
    \item PREM
    \item CRUST 2.0
  \end{itemize}

  Data converted to web browser consumable format (JSON)

  Internal data model?

  Event based nature of web browsers and what happens when a user moves a control

  The maths done when user inputs.

Technologies used:
\begin{itemize}
  \item Languages used:
  \begin{itemize}
    \item Python
    \item JavaScript
  \end{itemize}
  \item  Libraries Used
  \begin{itemize}
    \item D3
    \item JQuery
    \item Bootstrap
  \end{itemize}
  \item Modern Web Technologies:
  \begin{itemize}
    \item SVG (Color bars, map overlays)
    \item HTML5 Canvas (Actual Map)
  \end{itemize}
\end{itemize}
\section{Discussion}
Discuss potential researcher benifit.

Discuss potential education benifit.

Discuss future (obligatory additional research required)
\section{Acknowledgments}
HPU, CIDER
\bibliographystyle{plain}
\bibliography{refs}
\end{document}

